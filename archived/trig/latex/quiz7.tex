
\documentclass[12pt]{article}

% Packages for page layout
\usepackage[margin=1in]{geometry}
\usepackage{fancyhdr}

% Packages for math
\usepackage{amsmath, amsthm, amssymb, amsfonts}

% Packages for graphics
\usepackage{graphicx}
\usepackage{float}

% Header and footer
\pagestyle{fancy}
\fancyhf{}
\rhead{Rex's Trigonometry Problem Set}
\lhead{}
\rfoot{Page \thepage}

% Title page
\title{
    \vspace{2in}
    \textbf{Trigonometry Problem Set}\\
    \vspace{0.1in}
    \large for Class\\
    \vspace{3in}
}
\author{Rex}
\date{}

\begin{document}

\maketitle
\newpage

% Begin enumeration for problems
\begin{enumerate}
    % Problem 1
    \item \textbf{Solve the following equation for \( x \), where \( 0 \leq x < 2\pi \):}
    \[ 3\cos(2x) + 4\cos(x) - 7 = 0 \]
    
    \textit{Separate multiple solutions by commas. Enter \( \emptyset \) if there is no solution.}
    % Solution
    The solution to the equation is:
    \[ x = 0 \]
    % Space for the solution
    \vspace{45mm}
    % Solution goes here

     % Problem 2
    \item \textbf{If \( 0 \leq \alpha < 2\pi \), find all values of \( \alpha \) that satisfy the equation}
    \[ 3\cos(2\alpha) + 11\cos(\alpha) + 7 = 0 \]
    
    \textit{Separate multiple solutions by commas. Enter \( \emptyset \) if there is no solution.}
    
    % Solution
    The solutions to the equation are:
    \[ \alpha = \frac{2\pi}{3}, \frac{4\pi}{3} \]
    % Space for the solution
    \vspace{45mm}
    % Solution goes here
   \newpage 
    % Problem 3
    \item \textbf{Solve the following equation for \( \theta \) on the interval \( [0,2\pi) \):}
    \[ 5\sqrt{3} \tan(\theta) - 4 = 1 \]
    
    \textit{List the angles separated by commas if there are multiple answers, e.g. \( \frac{\pi}{3}, \frac{\pi}{2} \).}
    
    % Solution
    The solutions to the equation are:
    \[ \theta = \frac{\pi}{6}, \frac{7\pi}{6} \]
    % Space for the solution
    \vspace{45mm}
    % Solution goes here

    % Problem 4
    \item \textbf{Determine the exact value of \( \theta \) in the following equation if \( 0 \leq \theta < 2\pi \). Enter your answer separated by commas.}
    \[ -4\cos(\theta) + 5 = 5 \]
    
    \textit{List the angles separated by commas if there are multiple answers.}
    
    % Solution
    The solutions to the equation are:
    \[ \theta = \frac{\pi}{2}, \frac{3\pi}{2} \]
    % Space for the solution
    \vspace{45mm}
    % Solution goes here
   \newpage 
    % Problem 5
    \item \textbf{What are the amplitude and period of the function \( f(x) = 3\sin(-9x) \)?}
    
    % Solution
    The amplitude of the function is:
    \[ \text{Amplitude} = 3 \]
    
    The period of the function is:
    \[ \text{Period} = \frac{2\pi}{9} \]
    % Space for the solution
    \vspace{45mm}
    % Solution goes here
    
    % Problem 6 and 7
    \item \textbf{Solve for \( \theta \) if \( 16\sin(\theta) + 11 = 27 \) and \( 0 \leq \theta < 2\pi \).}
    
    % Solution
    The solution to the equation is:
    \[ \theta = \frac{\pi}{2} \]
    % Space for the solution
    \vspace{45mm}
    % Solution goes here

    % Problem 8
    \item \textbf{Solve for \( \theta \) if \( 2\cos(\theta) + 8 = 10 \) and \( 0 \leq \theta < 2\pi \). Enter your answer(s) in radians. If necessary, separate multiple values by commas.}
    
    % Solution
    The solution to the equation is:
    \[ \theta = 0 \]
    % Space for the solution
    \vspace{45mm}
    % Solution goes here

    
    % Problem 9
    \item \textbf{Solve the following equation for \( \theta \) on the interval \( [0,2\pi) \):}
    \[ -7\sqrt{3} \tan(\theta) + 2 = 9 \]
    
    \textit{Enter your answer(s) in radians. If necessary, separate multiple values by commas.}
    
    % Solution
    The solutions to the equation are:
    \[ \theta = \frac{5\pi}{6}, \frac{11\pi}{6} \]
    % Space for the solution
    \vspace{45mm}
    % Solution goes here

    
    
    % Problem
    \item \textbf{Solve the following equation for \( \theta \) on the interval \( [0,360^\circ) \):}
    \[ -5\sec(\theta) - 4 = -14 \]
    
    \textit{Select all correct answers.}
    
    \begin{itemize}
        \item \( 300^\circ \)
        \item \( 30^\circ \)
        \item \( 0^\circ \)
        \item \( 60^\circ \)
        \item \( 135^\circ \)
        \item \( 120^\circ \)
    \end{itemize}
    
    % Solution
    The solutions to the equation are:
    \[ \theta = 60^\circ, 300^\circ \]
    % Space for the solution
    \vspace{45mm}
    % Solution goes here

\end{enumerate}
\end{document}

