
\documentclass{article}
\usepackage[utf8]{inputenc}
\usepackage{graphicx}  % For including images
\usepackage{geometry}  % For adjusting margins and spacings
\usepackage{enumitem}
\usepackage{float}
\usepackage{hyperref}

\usepackage{fancyhdr}
\pagestyle{fancy}
\fancyhf{} % Clear all headers and footers
\renewcommand{\headrulewidth}{0.4pt} % Add a rule below the header

\title{Agile Development}

\lhead{Name: Christopher Nowacki}
\chead{Class: CIS 280}
\rhead{Assignment: Agile Development}
% Adjust page margins
\geometry{top=2.5cm, bottom=2cm, left=2cm, right=2cm, headheight=15pt}

\begin{document}
\thispagestyle{fancy}

\section{Agile Development Assignment}

\subsection*{Part I}

\textbf{Write a brief history of Agile development. Who started it? When? What are its main characteristics? What separates Agile techniques from other iterative life cycle techniques? What are its advantages \& disadvantages?}

Agile development is a methodology that has become overwhelmingly popular in the world of project management/software development. Agile officially launched in 2001 with the release of the \textit{Agile Manifesto}. There are four major principles to agile project management:
\begin{enumerate}
    \item Individuals and interactions over processes and tools
    \item Working software over comprehensive documentation
    \item Customer collaboration over contract negotiation
    \item Responding to change over following a plan
\end{enumerate}

Agile differs from other iterative life cycle techniques in that it is more focused on adaptability versus rigidity, it involves customer feedback and interaction throughout the development cycle, and there is more emphasis on people and collaboration rather than rigid processes and tooling.

It has many advantages -- it is flexible, meaning it can adapt to changing requirements. The fact that customers are continuously involved leads to greater satisfaction outcomes and allows the project manager to identify and address project risks early. The biggest advantage is how it allows for a faster release schedule.

There are also several disadvantages -- for instance -- its less predictable nature makes it less predictable which can lead to scope creep and difficulties in regards to timelines and budgets. The fact that it relies on continuous interactions from the customer can also cause difficulties. It may also not be the right fit for certain teams who are not naturally collaborative.

\subsection*{Part II}

\textbf{Scrum is an iterative and incremental Agile framework for managing software product development. Describe the following:}
\begin{itemize}
    \item The history of Scrum
    \item Each of the roles \& responsibilities in the Scrum process
    \item The Sprint
    \begin{itemize}
        \item What are the activities in a Sprint?
        \item Are there meetings?
        \item How do you know when a Sprint is done?
        \item What Artifacts are produced during a Sprint?
        \item The Product Backlog \& Product Backlog Items
    \end{itemize}
    \item Scrum terminology, like Scrum-But, Tracer Bullet, \& Sashimi.
\end{itemize}

The idea for Scrum was taken from a 1986 Harvard Business Review article called "The New New Product Development Game" by Takeuchi and Nonaka. The article was about the effectiveness of cross-functional teams developing products via an "everybody-in" approach.\\ 

It was adapted to the software development business in 1993 by Jeff Sutherland and Ken Schwaber. They would go on to publish several papers and books, including "The Scrum Guide", which outlined the framework. 

\textbf{Roles \& Responsibilities in Scrum}
\begin{enumerate}
    \item Product Owner: Responsible for maximizing product value; they manage the Product Backlog and make sure it is transparent and clear.
    \item Scrum Master: Facilitator for the team and Product Owner. They help everyone understand Scrum theory, practices, and rules. 
    \item Dev Team: Do the work of delivering a releasable increment of "done" product at the end of each sprint. 
\end{enumerate}

A sprint is a specified period of time during which a "done" potentially usable product is created.\\\\
\textbf{Activities \& Meetings}:
\begin{enumerate}
    \item Sprint Planning -- determine what will be delivered and how to achieve it. 
    \item Daily Scrum: A 15-minute timed event for the Dev team to synchronize activities and create a plan for the next 48 hours.
    \item Sprint Review: Held at the end of the sprint to review the "increment".
    \item Sprint Retrospective: The team discusses what went right and what went wrong and what will be committed to the next sprint. 
\end{enumerate}

The sprint is done when the agreed upon time period is over (usually 2-4 weeks) and the Sprint Goal is achieved. 
\\\\
\textbf{Artifacts Produced}:
\begin{enumerate}
    \item Product Increment: The sum of all the Product Backlog items completed during a Sprint and all previous sprints.
    \item Sprint Backlog: A set of items selected for the sprint, plus a plan for delivering the product increment and achieving the Sprint Goal. 
\end{enumerate}

The product Backlog is an \textbf{ordered list} of everything that is known to be needed for the product. Product Backlog items are the individual tasks or requirements that make up the backlog.
\\\\
\textbf{Scrum-But}: a situation where a team is using Scrum, but with modifications that may or may not lead to inefficiency.
\\\textbf{Tracer Bullet}: A development approach to explore unknown areas of the project.
\\\textbf{Sashimi}: describes a way of ensuring that every piece of work is fully developed and tested.

\subsection*{Part III}

\textbf{Many development teams are using hybrid versions of Scrum, which Scrum purists say isn't using Scrum at all. Find a company that uses Agile development techniques, \& describe their situation. What motivated them to switch to Agile/Scrum? Are they happy with it? Has it been a success or a failure?}

Many successful software companies use Scrum techniques, but it has become very common to use Scrum where it fits and to discard it in other places. Netflix, for example, uses a hybrid version of Scrum. Netflix adopted Agile due to the fact that it needed to respond quickly to market changes and technological advancements. Agile allowed them to increase the speed of their development cycle and quickly improve the quality of their product. The speed at which they did so was pivotal in their success, as they were one of the first to have a beautifully intuitive UI, advanced algorithms for searches and providing user recommendations, etc. 

While they do not strictly adhere to Scrum methods, Netflix has been happy with their hybrid Agile approach -- it has allowed them to maintain their position as a leader in streaming and they have become an "IT" company that people desperately want to work at.

\end{document}
