
\documentclass[12pt]{article}

% Packages for page layout
\usepackage[margin=1in]{geometry}
\usepackage{fancyhdr}
\usepackage{enumitem} % Custom list labels

% Packages for math
\usepackage{amsmath, amsthm, amssymb, amsfonts}

% Packages for graphics
\usepackage{graphicx}
\usepackage{float}

% Header and footer
\pagestyle{fancy}
\fancyhf{}
\rhead{Quiz 11: OH MY GOD}
\lhead{}
\rfoot{Page \thepage}

% Title page
\title{
    \vspace{2in}
    \textbf{Quiz 10: Solving Trigonometric Equations, Pythagorean Identity, and Verifying Trig Identities}\\
    \vspace{0.1in}
    \large There's a Ghost in My Calculator\\
    \vspace{3in}
}
\author{Cthulhu Lemon}
\date{}

\begin{document}

\maketitle
\newpage

% Begin enumeration for problems
\begin{enumerate}
    % Problem 1
    \item \textbf{If \( \sin(\theta) = \frac{16}{34} \) and \( \theta \) is in Quadrant II, what is \( \sin\left(\frac{\theta}{2}\right) \)?} Type an exact answer, using radicals as needed. Simplify your answer completely and rationalize the denominator.
    \vspace{45mm}
    % Solution goes here
    % Problem 2

    \item \textbf{Solve the equation \( 2\cos(2x) + 12\cos(x) + 7 = 0 \) on the interval \( 0 \leq x < 2\pi \).} Enter an exact answer in terms of \(\pi\).
    \vspace{45mm}
    % Solution goes here
    % Problem 3
    \item \textbf{Given that \( \sin(\theta) = \frac{12}{13} \), and \( \theta \) is in Quadrant II, what is \( \cos(2\theta) \)?} Give an exact answer in the form of a fraction.
    \vspace{45mm}
    % Solution goes here

    % Problem 4
    
    \item \textbf{You are given that \( \cos(A) = \frac{15}{17} \), with \( A \) in Quadrant IV, and \( \cos(B) = \frac{4}{5} \), with \( B \) in Quadrant IV. Find \( \cos(A - B) \).} Give your answer as a fraction.
    \vspace{45mm}
    % Solution goes here
    % Problem 5

    \item \textbf{For all values of \( \theta \) for which the expression is defined, what is \( \frac{\cos(2\theta)}{\tan^2\theta} \)?}
    \vspace{45mm}
    % Solution goes here
    % Problem 6
    % Continue from the previous problems

    \item \textbf{For all values of \( \alpha \) and \( \beta \) for which the expression is defined, what is \( \frac{\cos(\alpha+\beta)}{\sin \beta} \)?}
    \vspace{45mm}
    % Solution goes here

    \item \textbf{Compute the exact value of \( \tan\left(\frac{17\pi}{12}\right) \).}
    \vspace{45mm}
    % Solution goes here

    \item \textbf{If \( 0 \leq x < 2\pi \), find all values of \( x \) that satisfy the equation}
    \[ \cos^3\left(-\frac{x}{2}\right)\tan^2 x + \cos^3\left(-\frac{x}{2}\right) = \frac{\sqrt{3}}{2} \]
    \textit{Enter an exact answer, in terms of \( \pi \).}
    \vspace{45mm}
    % Solution goes here

    \item \textbf{Compute the exact value of \( \sin\left(\frac{25\pi}{12}\right) \).}
    \vspace{45mm}
    % Solution goes here

    \item \textbf{Solve the following equation for \( x \), where \( 0 \leq x < 2\pi \):}
    \[ \cos(2x) + 8\cos x - 9 = 0 \]
    \vspace{45mm}
    % Solution goes here
\end{enumerate}
\end{document}
